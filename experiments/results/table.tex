\documentclass{article}
\usepackage{amssymb}
\usepackage{booktabs}
\usepackage{subcaption}
\usepackage{colortbl}
\usepackage[dvipsnames]{xcolor}
\usepackage{graphicx}
\captionsetup[subfigure]{position=bottom}
\begin{document}
\newcommand{\haliver}{HaliVer}
\begin{table}[t]
\centering
\caption{\label{tab:chp6-results-exp-mem}Verification results for the experiments of HaliVer from Chapter 4.}
\begin{tabular}{lll|rrr|rrr|r}
\hline
 & & & \multicolumn{3}{c|}{\textbf{Base}} & \multicolumn{3}{c|}{\textbf{Unique}} & \\
\textbf{Name} & \textbf{V} & \textbf{Result} & \textbf{\#} & \textbf{T$_t$} & \textbf{T$_v$} & \textbf{\#} & \textbf{T$_t$} & \textbf{T$_v$} & \textbf{Speedup$_v$} \\
\hline
blur\ & 0& \checkmark&  5& 31 &  8&  5& 30 &  7& \cellcolor{ForestGreen!25} 1.14 \\
\hline
 & 1& \checkmark&  5& 32 &  9&  5& 30 &  8& \cellcolor{ForestGreen!25} 1.12 \\
\hline
 & 2& \checkmark&  5& 36 &  11&  5& 33 &  9& \cellcolor{ForestGreen!25} 1.22 \\
\hline
 & 3& \checkmark&  5& 35 &  11&  5& 33 &  9& \cellcolor{ForestGreen!25} 1.22 \\
\hline
hist\ & 0& \checkmark&  4& 42 &  16&  5& 37 &  11& \cellcolor{ForestGreen!25} 1.45 \\
& & $\times$& 0 &  & & 1 & 101 & 76 \\
\hline
 & 1& \checkmark&  5& 48 &  21&  5& 39 &  13& \cellcolor{ForestGreen!25} 1.62 \\
\hline
 & 2& \checkmark&  5& 72 &  45&  5& 44 &  17& \cellcolor{ForestGreen!25} 2.65 \\
\hline
 & 3& \checkmark&  4& 103 &  74&  4& 46 &  19& \cellcolor{ForestGreen!25} 3.89 \\
& & $\times$& 1 & 100 & 74& 1 & 146 & 120 \\
\hline
conv\_layer\ & 0& \checkmark&  5& 118 &  86&  5& 70 &  40& \cellcolor{ForestGreen!25} 2.15 \\
\hline
 & 1& \checkmark&  5& 134 &  101&  5& 72 &  42& \cellcolor{ForestGreen!25} 2.4 \\
\hline
 & 2& \checkmark&  5& 196 &  159&  5& 75 &  44& \cellcolor{ForestGreen!25} 3.61 \\
\hline
 & 3& \checkmark&  4& 174 &  138&  5& 75 &  44& \cellcolor{ForestGreen!25} 3.14 \\
& & $\times$& 0 &  & & 1 & 130 & 96 \\
\hline
gemm\ & 0& \checkmark&  5& 59 &  32&  5& 40 &  15& \cellcolor{ForestGreen!25} 2.13 \\
\hline
 & 1& \checkmark&  5& 94 &  62&  5& 51 &  23& \cellcolor{ForestGreen!25} 2.7 \\
\hline
 & 2& \checkmark&  5& 133 &  98&  5& 70 &  40& \cellcolor{ForestGreen!25} 2.45 \\
\hline
 & 3& $\times$& 5 & 59 & 26& 5 & 119 & 80 \\
\hline
auto\_viz\ & 0& \checkmark&  5& 46 &  15&  5& 41 &  12& \cellcolor{ForestGreen!25} 1.25 \\
\hline
 & 1& \checkmark&  5& 97 &  68&  5& 50 &  21& \cellcolor{ForestGreen!25} 3.24 \\
\hline
 & 2& \checkmark&  5& 98 &  67&  5& 52 &  22& \cellcolor{ForestGreen!25} 3.05 \\
\hline
 & 3& \checkmark&  5& 73 &  39&  5& 54 &  22& \cellcolor{ForestGreen!25} 1.77 \\
\hline
\multicolumn{2}{l}{bilateral\_grid}& \checkmark&  5& 77 &  41&  5& 63 &  28& \cellcolor{ForestGreen!25} 1.46 \\
\hline
\multicolumn{2}{l}{camera\_pipe}& \checkmark& \cellcolor{BrickRed!25} 0&  &  & \cellcolor{ForestGreen!25} 2& 304 &  264&  \\
& & $\times$& 3 & 435 & 397& 4 & 3084 & 3044 \\
& & T.O.& 0 & - & -& 1 & - & - \\
\hline
\multicolumn{2}{l}{depthwise\_}& \checkmark&  5& 214 &  165&  5& 139 &  96& \cellcolor{ForestGreen!25} 1.72 \\
separable\_conv & & & & & & & & \\
\hline
\hline
Total & & \checkmark & & 1912 & 1266 & & 1144 & 542 & \cellcolor{ForestGreen!25} 2.34  \\
\hline
\end{tabular}
\end{table}
\begin{table}[t]
\centering
\caption{\label{tab:chp6-results-exp}Verification results for the experiments of HaliVer from Chapter 4.}
\begin{tabular}{lll|rrr|rrr|r}
\hline
 & & & \multicolumn{3}{c|}{\textbf{Base}} & \multicolumn{3}{c|}{\textbf{Unique}} & \\
\textbf{Name} & \textbf{V} & \textbf{Result} & \textbf{\#} & \textbf{T$_t$} & \textbf{T$_v$} & \textbf{\#} & \textbf{T$_t$} & \textbf{T$_v$} & \textbf{Speedup$_v$} \\
\hline
blur\ & 0& \checkmark&  5& 33 &  10&  5& 33 &  8& \cellcolor{ForestGreen!25} 1.25 \\
\hline
 & 1& \checkmark&  5& 33 &  10&  5& 32 &  8& \cellcolor{ForestGreen!25} 1.25 \\
\hline
 & 2& \checkmark&  5& 43 &  14&  5& 42 &  14&  1.0 \\
\hline
 & 3& \checkmark&  5& 42 &  15&  5& 56 &  29& \cellcolor{BrickRed!25} 0.52 \\
\hline
hist\ & 0& \checkmark&  4& 46 &  18&  5& 38 &  12& \cellcolor{ForestGreen!25} 1.5 \\
& & $\times$& 1 & 287 & 260& 0 &  &  \\
\hline
 & 1& \checkmark&  5& 54 &  25&  5& 41 &  14& \cellcolor{ForestGreen!25} 1.79 \\
\hline
 & 2& \checkmark&  5& 81 &  53&  5& 45 &  18& \cellcolor{ForestGreen!25} 2.94 \\
\hline
 & 3& \checkmark&  4& 108 &  78&  4& 52 &  23& \cellcolor{ForestGreen!25} 3.39 \\
& & $\times$& 1 & 170 & 136& 1 & 106 & 77 \\
\hline
conv\_layer\ & 0& \checkmark&  5& 127 &  91&  5& 70 &  41& \cellcolor{ForestGreen!25} 2.22 \\
\hline
 & 1& \checkmark&  5& 142 &  106&  5& 74 &  44& \cellcolor{ForestGreen!25} 2.41 \\
\hline
 & 2& \checkmark&  4& 222 &  182&  5& 78 &  46& \cellcolor{ForestGreen!25} 3.96 \\
& & $\times$& 1 & 134 & 97& 0 &  &  \\
\hline
 & 3& \checkmark&  5& 186 &  148&  5& 78 &  46& \cellcolor{ForestGreen!25} 3.22 \\
\hline
gemm\ & 0& \checkmark&  5& 63 &  35&  5& 41 &  16& \cellcolor{ForestGreen!25} 2.19 \\
\hline
 & 1& \checkmark&  5& 118 &  82&  5& 69 &  38& \cellcolor{ForestGreen!25} 2.16 \\
\hline
 & 2& \checkmark&  5& 266 &  225&  4& 187 &  151& \cellcolor{ForestGreen!25} 1.49 \\
& & $\times$& 0 &  & & 1 & 112 & 76 \\
\hline
 & 3& $\times$& 5 & 73 & 27& 5 & 55 & 15 \\
\hline
auto\_viz\ & 0& \checkmark&  5& 104 &  72&  5& 89 &  59& \cellcolor{ForestGreen!25} 1.22 \\
\hline
 & 1& \checkmark&  5& 172 &  141&  5& 76 &  44& \cellcolor{ForestGreen!25} 3.2 \\
\hline
 & 2& \checkmark&  5& 174 &  139&  5& 75 &  42& \cellcolor{ForestGreen!25} 3.31 \\
\hline
 & 3& \checkmark&  5& 104 &  66&  5& 71 &  35& \cellcolor{ForestGreen!25} 1.89 \\
\hline
\hline
Total & & \checkmark & & 2118 & 1510 & & 1247 & 688 & \cellcolor{ForestGreen!25} 2.19  \\
\hline
\end{tabular}
\end{table}
\begin{table}[t]
\centering
\caption{\label{tab:chp6-results}Verification results for \texttt{step}, \texttt{sub\_direction}, \texttt{solve\_direction}, and \texttt{perform\_iteration} produced by \haliver. We use abbreviations for versions with concrete bounds (\textbf{CB}), nonconcrete bounds (\textbf{NCB}), \textbf{unique} and const type qualifiers, and no type qualifiers (\textbf{Normal}).}

\subfloat[\label{tab:StepHalide}\texttt{\texttt{step}}]{
\resizebox{\widthPadre\textwidth}{!}{
\begin{tabular}{ll|rrr|rrr|r}
\hline
 & & \multicolumn{3}{c|}{\textbf{Base}} & \multicolumn{3}{c}{\textbf{Unique}} & \\
\textbf{Version} & \textbf{Result} & \textbf{\#} & \textbf{T$_t$} & \textbf{T$_v$} & \textbf{\#} & \textbf{T$_t$} & \textbf{T$_v$} & \textbf{Speedup$_v$} \\
\hline
CB& \checkmark&  5& 73 &  41&  5& 65 &  34& \cellcolor{ForestGreen!25} 1.21 \\
\hline
NCB& \checkmark&  5& 75 &  41&  5& 64 &  34& \cellcolor{ForestGreen!25} 1.21 \\
\hline
\end{tabular}
}
}
\\
\subfloat[\label{tab:SubDirectionHalide}\texttt{\texttt{sub\_direction}}]{
\resizebox{\widthPadre\textwidth}{!}{
\begin{tabular}{ll|rrr|rrr|r}
\hline
 & & \multicolumn{3}{c|}{\textbf{Base}} & \multicolumn{3}{c}{\textbf{Unique}} & \\
\textbf{Version} & \textbf{Result} & \textbf{\#} & \textbf{T$_t$} & \textbf{T$_v$} & \textbf{\#} & \textbf{T$_t$} & \textbf{T$_v$} & \textbf{Speedup$_v$} \\
\hline
CB& \checkmark&  5& 270 &  229&  5& 165 &  127& \cellcolor{ForestGreen!25} 1.8 \\
\hline
NCB& \checkmark& \cellcolor{BrickRed!25} 0&  &  & \cellcolor{ForestGreen!25} 5& 165 &  128&  \\
& $\times$& 0 &  & & 4 & 702 & 660 \\
& T.O.& 0 & - & -& 1 & - & - \\
\hline
\end{tabular}
}
}
\\
\subfloat[\label{tab:SolveDirectionHalide}\texttt{\texttt{solve\_direction}}]{
\resizebox{\widthPadre\textwidth}{!}{
\begin{tabular}{ll|rrr|rrr|r}
\hline
 & & \multicolumn{3}{c|}{\textbf{Base}} & \multicolumn{3}{c}{\textbf{Unique}} & \\
\textbf{Version} & \textbf{Result} & \textbf{\#} & \textbf{T$_t$} & \textbf{T$_v$} & \textbf{\#} & \textbf{T$_t$} & \textbf{T$_v$} & \textbf{Speedup$_v$} \\
\hline
CB& \checkmark& \cellcolor{BrickRed!25} 0&  &  & \cellcolor{ForestGreen!25} 5& 1022 &  805&  \\
& $\times$& 0 &  & & 5 & 1555 & 1343 \\
\hline
NCB& $\times$& 3 & 3072 & 2792& 5 & 925 & 631 \\
& T.O.& 2 & - & -& 0 & - & - \\
\hline
\end{tabular}
}
}
\\
\subfloat[\label{tab:PerformIterationHalide}\texttt{\texttt{perform\_iteration}}]{
\resizebox{\widthPadre\textwidth}{!}{
\begin{tabular}{ll|rrr|rrr|r}
\hline
 & & \multicolumn{3}{c|}{\textbf{Base}} & \multicolumn{3}{c}{\textbf{Unique}} & \\
\textbf{Version} & \textbf{Result} & \textbf{\#} & \textbf{T$_t$} & \textbf{T$_v$} & \textbf{\#} & \textbf{T$_t$} & \textbf{T$_v$} & \textbf{Speedup$_v$} \\
\hline
CB& \checkmark& \cellcolor{BrickRed!25} 0&  &  & \cellcolor{ForestGreen!25} 5& 1198 &  1033&  \\
& $\times$& 0 &  & & 4 & 2008 & 1846 \\
& T.O.& 0 & - & -& 1 & - & - \\
\hline
NCB& $\times$& 4 & 3062 & 2861& 5 & 2265 & 2049 \\
& T.O.& 1 & - & -& 0 & - & - \\
\hline
\end{tabular}
}
}
\\
\end{table}
\end{document}